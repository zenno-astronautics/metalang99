\documentclass[a4paper, 12pt]{article}

\usepackage{hyperref}
\usepackage{float}
\usepackage{bussproofs}
\usepackage{minted}
\usepackage{amsmath}
\usepackage{geometry}

\geometry{margin=60px}
\floatstyle{boxed}
\restylefloat{figure}

\begin{document}

\title{Macrolop Specification}
\date{\today}
\author{Temirkhan Myrzamadi (a.k.a. Hirrolot)}
\maketitle

\tableofcontents

\newpage

\section{EBNF Grammar}

\begin{figure}[h]
    \caption{Grammar rules}

\begin{minted}{bnf}
<eval> ::= "MACROLOP_EVAL(" { <term> }* ")" ;

<term> ::= "call(" <op> "," { <term> }* ")"
         | "v(" <preprocessor-token-list> ")" ;

<op>   ::= <ident> | { <term> }+ ;
\end{minted}

\end{figure}

A metaprogram in Macrolop consists of a (possibly empty) sequence of terms, each of which
is either a macro call or just a value.

Notes:

\begin{itemize}
    \item The grammar above describes metaprograms already expanded by the C preprocessor,
    except for \texttt{MACROLOP\_EVAL}, \texttt{call}, and \texttt{v}.
    \item \texttt{call} accepts \texttt{op} either as an identifier or as a non-empty
    sequence of terms that reduces to an identifier.
    \item \texttt{call} accepts arguments without a separator.
\end{itemize}

\section{Operational Semantics}

We define small-step operational semantics for Macrolop.

\begin{figure}[h]
    \caption{Computational rules}

    \begin{align*}
        (v): \ & \langle k; acc; v(\overline{tok}) \ term \ \overline{term'} \rangle & \to_1 &
            \langle k; acc, \ \overline{tok}; term \ \overline{term'} \rangle \\
        (v\mbox{-}end): \ & \langle k; acc; v(\overline{tok}) \rangle & \to_1 &
            k(seq\mbox{-}extract(acc, \overline{tok})) \\
        (op): \ & \langle k; acc; call(\overline{term}, \overline{a}) \ \overline{term'} \rangle & \to_1 &
            \langle \langle k; acc; call(?, \overline{a}) \ \overline{term'} \rangle; (); \overline{term} \rangle \\
        (args): \ & \langle k; acc; call(ident, \overline{a}) \ \overline{term} \rangle & \to_1 &
            \langle \langle k; acc; ident(?) \ \overline{term} \rangle; (); \overline{a} \rangle \\
        (start): \ & MACROLOP\_EVAL(\overline{term}) & \to_1 &
            \langle halt; (); \overline{term} \rangle \\
    \end{align*}
\end{figure}

Notes:

\begin{itemize}
    \item A body of a macro called using \texttt{call} must follow the grammar of
    Macrolop, otherwise it might result in a compilation error.
\end{itemize}

\end{document}
